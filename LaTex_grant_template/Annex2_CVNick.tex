\documentclass[10pt]{article}
\usepackage{charter}
\usepackage{libertine}
\usepackage{hyperref}
\hypersetup{
    bookmarks=false,         % show bookmarks bar?
    unicode=false,          % non-Latin characters in Acrobats bookmarks
    pdftoolbar=false,        % show Acrobats toolbar?
    pdfmenubar=true,        % show Acrobats menu?
    pdffitwindow=false,     % window fit to page when opened
    pdfstartview={FitH},    % fits the width of the page to the window
    pdftitle={},    % title
    pdfauthor={Daria Shipilina},     % author
    pdfsubject={},   % subject of the document
    pdfcreator={},   % creator of the document
    pdfproducer={}, % producer of the document
    pdfkeywords={ }{}, % list of keywords
    pdfnewwindow=true,      % links in new window
    colorlinks=true,       % false: boxed links; true: colored links
    linkcolor=black,          % color of internal links (change box color with linkbordercolor)
    citecolor=blue,        % color of links to bibliography
    filecolor=magenta,      % color of file links
    urlcolor=blue           % color of external links
}
% biblio-ref 
%\usepackage[round]{natbib}
%\bibliographystyle{plainnat}

\usepackage{fullpage}
\usepackage[resetlabels]{multibib}
\usepackage[ManyBibs]{currvita}
%\usepackage[dvips]{color}
%\usepackage[T2A]{fontenc}
%\usepackage[english,russian]{babel}
\usepackage{amsfonts,amssymb,graphicx}
%\usepackage{multicol}
\usepackage[rflt]{floatflt}
% Initialize each paper type for which you need a bibliography.
% Just a dummy parameter is necessary.
\usepackage{lastpage,fancyhdr}
\date{}
\newcites{refpaper}{foo}
\newcites{unrefpaper}{foo}
\newcites{journal}{foo}
\newcites{techreport}{foo}
\newcites{submitpaper}{foo}
\newcites{classproject}{foo}

% Better for lists with 1-2 items and short descriptions
\newenvironment{sublist}{%
	\begin{list}{}{%
		\setlength{\itemsep}{0em}\setlength{\parsep}{0em}%
		\setlength{\topsep}{0em}\setlength{\parskip}{0em}%
	}%
}%
{ \end{list} }

% Better for lists with more than 2 items and/or long descriptions
\newenvironment{subbulletlist}{%
	\begin{list}{\labelitemii}{%
		\setlength{\topsep}{\itemsep}\setlength{\parskip}{\parsep}%
	}%
}%
{ \end{list} }
%Fancy package for page enumeration and headers
\pagestyle{fancy}
%\pagestyle{empty}
%\cfoot{Page \thepage\ of \pageref{LastPage}}
\setcounter{page}{23}
\cfoot{\thepage\ }
%\cfoot{}
\lhead{\Large \textbf {Nicholas H. Barton}}
\chead{{\textbf{\large \itshape Curriculum Vitae  }}}%Curriculum %Vitae 
\voffset=-1.cm
%\textheight = 670pt
%\footskip = 60pt
%\headheight=18pt
\textheight = 670pt
%\voffset=1.25cm
\headsep=26pt
\footskip = 25pt
\begin{document}
% Specify the bibtex style that you want for each paper type
\bibliographystyle{unsrt}
\bibliographystylerefpaper{unsrt}
\bibliographystyleunrefpaper{unsrt}
\bibliographystylesubmitpaper{unsrt}
\bibliographystylejournal{unsrt}
\bibliographystyletechreport{unsrt}
\bibliographystyleclassproject{ieeetr}

% We'll use this length to change the defaults in some of our lists.
\newlength{\oldcvlabelwidth}
\renewcommand*{\cvbibname}{}

% This is what will appear at the very top of your CV.



\begin{cv}



\begin{cvlist}{Contact}
    \item Am Campus 1\\
     Klosterneuburg, Austria 3400\\
      phone: +43 2243 9000-3001\\
      e-mail: nick.barton@ist.ac.at\\%daria.sh@berkeley.edu
      web-site: ist.ac.at/research-group-pages/barton-group
\end{cvlist}

\begin{cvlist}{Main Areas of Research}
\item My focus has been to understand \textbf{hybrid zones}: narrow regions in which distinct populations meet, mate and produce hybrids.  I have worked on a variety of field systems, including grasshoppers (\textit{Podisma pedestris}), toads (\textit{Bombina bombina/variegata}), butterflies (\textit{Heliconius erato} and \textit{H.melpomene}), and snapdragons (\textit{Antirrhinum majus}).  This led naturally to development of \textbf{mathematical models} for how populations evolve when they extend through space, and when large numbers of genes interact.  These models have been applied to a wide range of questions, and in particular, to \textbf{speciation}, and to the  \textbf{quantitative genetics} of complex traits.  Key components of my present work are: i) a long-term study of the Antirrhinum hybrid zone, and ii) development of methods for inferring population structure and selection from DNA sequence variation, from both natural and experimental populations.
\end{cvlist}

\begin{cvlist}{Academic Career}
\item[2015 -- present] \textbf{Dean of Graduate School}\\
\textit{Institute of Science and Technology Austria (IST)}, Austria
\item[2008 -- present] \textbf{Professor}, \textit{Institute of Science and Technology (IST)}, Austria
\item[2000 ‐- 2010] \textbf{Personal Chair}, Institute of Evolutionary Biology\\
 \textit{University of Edinburgh}, UK
\item[1990 ‐- 2000]  \textbf{Darwin Trust Fellow}, Institute of Evolutionary Biology\\
 \textit{University of Edinburgh}, UK
\item[1982 ‐- 1990]  \textbf{Lecturer/Reader}, Department of Genetics, 
 \textit{University College London}, UK
\item[1979 ‐- 1982] \textbf{Research Fellowship},  \textit{Girton College}, UK
\item[1976 -- 1979]   \textbf{Ph.D. in Biology} (Supervised by Dr. G.M. Hewitt)\\
       \textit{University of East Anglia}, UK
        \begin{sublist}
     		\item Thesis:~A narrow hybrid zone in the alpine grasshopper \textit{Podisma pedestris}
	       \end{sublist}
\item[1973 ‐- 1976] \textbf{B.A. (First Class) in Natural Sciences (Genetics)}\\
              \textit{University of Cambridge}, UK   
\end{cvlist}

\noindent{\bf \large Additional research achievements}
\begin{enumerate}
\item \textbf{Scientific awards}:  Erwin Schrodinger Prize (2013), \textit{Austrian Academy of Sciences}; Mendel Medal (2013),  \textit{Leopoldina}, Darwin‐Wallace Medal (2009), \textit{Linnean Society of London}; Darwin Medal (2006), \textit{The Royal Society}; Zoological Society Scientific Medal (1992); American Society of Naturalists President's Award (joint with Mark Kirkpatrick) (1998); David Starr Jordan Prize (joint with S. Pacala),(1994); Linnean Society Bicentennial Medal (1985). 

\item \textbf{Academic Memberships and community service}: Elected President, \textit{Society for the Study of Evolution} (2001, on Council 2000‐2002); Elected Vice‐President, \textit{Society for the Study of Evolution} (1989); Chair, \textit{Human Frontiers Fellowship panel}(2013-2016);         
Chair, \textit{ERC Starting Grants Panel LS8 }(2013 ‐- present); Elected President, \textit{Royal Society, Biological Sciences Awards Committee} (2009‐2011), Co‐Chair, \textit{Portuguese Foundation for Science and Technology (FCT)}, Chair, \textit{Royal Society Research Grants Committee (Board H)} (2004 ‐ 2006).

\item \textbf{Editorships}: Senior Editor, \textit{Genetics} (2016 -- present), Associate Editor, \textit{Molecular Ecology} 013 ‐- present], Associate Editor, \textit{Genetics} (2013 -- present), Editorial Board, \textit{Philosophical Transactions of the Royal Society B} (2010 ‐- 2015), Handling Editor, \textit{Evolution} (2008‐2011), Editorial Board, PLoS Biology (2004 ‐ 2007),  Editorial Board, \textit{American Naturalist} (2003),  Editorial Board, \textit{Journal of Evolutionary Biology} (1997‐2000).

\item \textbf{Recent Grants}: 2010‐2015  -- ERC Advanced Grant ``Information and Evolution”, \texteuro1.9M, 2010‐2014 -- Natural Selection in Spatially Structured Populations, (EPSRC 1013091 \texteuro215K), with Alison Etheridge, 2014‐2018 -- FP7 FET grant SAGE ``Speed of Adaptation in Population Genetics and Evolutionary Algorithms”, grant with T. Paixao (IST), Per Kristian Lehre (Nottingham), Tobias Friedrich (Jena), Dirk Sudholt (Sheffield). (\texteuro366K)

\end{enumerate} 


\begin{cvlist}{Publications}
\item
\end{cvlist}


\textbf{\quad 10 Most Important Scientific Publications}
            \begin{enumerate}%{sublist}{enumerate}	
\item Barton, N.H. 2017. How does epistasis influence the response to selection? Heredity 118.1 (2017): 96-109. \href{http://dx.doi.org/10.1038/hdy.2016.109}{doi:10.1038/hdy.2016.109}
\item Charlesworth, D., Barton, N.H., Charlesworth, B. 2017. The sources of adaptive evolution. Proc. Roy. Soc. (Lond.) B 284: 2016.2864. \href{http://dx.doi.org/doi: 10.1098/rspb.2016.2864}{doi:doi: 10.1098/rspb.2016.2864}
\item Ringbauer, H., Coop, G., Barton, N. H.. (2017). Inferring recent demography from isolation by distance of long shared sequence blocks. Genetics 205(3):1335-1351 \href{http://dx.doi.org/10.1534/genetics.116.196220}{doi:10.1534/genetics.116.196220}
\item Paixao, T., Barton, N.H. 2016. The effect of gene interactions on the long-term response to selection. PNAS 113: 4422-4427 \href{http://dx.doi.org/10.1073/pnas.1518830113}{doi:10.1073/pnas.1518830113}
\item Paixao, T., Badkobehe, G., Barton, N.H., Dolgan, C., Dang, D.C., Friedrich, T., Lehre, P.K., Sudholt, D., Trubenova, B. 2015. Towards a unifying framework for evolutionary processes. J. Theor. Biol. 383: 28-43.\\ \href{http://dx.doi.org/doi:10.1016/j.jtbi.2015.07.011}{doi:10.1016/j.jtbi.2015.07.011}
\item Barton, N.H. (2009) Why sex and recombination? Cold Spring Harbor Symposia Quant. Biol. 74 \\ \href{http://dx.doi.org/doi: 10.1101/sqb.2009.74.030}{doi: 10.1101/sqb.2009.74.030}
\item Barton, N.H., Briggs, D.E.G., Eisen, J.A., Goldstein, D.B., Patel, N.H. (2007) Evolution. Cold Spring Harbor Laboratory Press. ISBN-13: 9780879696849
\item Kirkpatrick, M., Johnson, T. Barton, N.H. (2002) General models of multilocus evolution. Genetics 161: 1727-1750 PMID: 12196414 
\item Coyne, J.A., N.H. Barton, and M. Turelli. (1997) A critique of Wright's shifting balance theory of evolution. Evolution 51: 643-671. \href{http://dx.doi.org/10.1111/j.1558-5646.1997.tb03650.x}{doi:10.1111/j.1558-5646.1997.tb03650.x}
\item Barton, N. H. (1995) Linkage and the limits to natural selection. Genetics 140: 821-841. PMID: 7498757
\end{enumerate}	


%\begin{cvlist}{Teaching Experience}

%\end{cvlist}

%\setlength{\oldcvlabelwidth}{\cvlabelwidth}
%\setlength{\cvlabelwidth}{6em}





% Reset the label indentation
%\newpage
\setlength{\cvlabelwidth}{\oldcvlabelwidth}


%\begin{cvlist}{Personal Information}
%    \item Date of birth:  August 21, 1988\\
          %Citizenship:  Russia \\
 %         Languages: English, Russian.
%\end{cvlist}
%\begin{cvlist}{Personal Information}
 %   \item Date of birth:  January 30th, 1986\\
 %         Citizenship:  Ukraine \\
 %         Languages: English (fluent), Russian (native), Ukrainian (native), German (basic)
%\end{cvlist}

%\begin{cvlist}{Personal Interests}
%\item  Reading, Travelling, Music
%\end{cvlist}
%\footskip = -10pt
%%%%%%%%%%%%%%%%%%%%%%%%%%%%%%%%%%%%%%%%%%%%%%%%%%%%%%%%%%%%%%%%%%%%%%%%%%%%%%%%%%%%%%%%%
%%%%%%%%%%%%%%%%%%%%%%%%%%%%%%%%%%%%%%%%%%%%%%%%%%%%%%%%%%%%%%%%%%%%%%%%%%%%%%%%%%%%%%%%%
\end{cv}
\end{document}

