\documentclass[10pt]{article}
\usepackage{charter}
\usepackage{fullpage}
\usepackage[resetlabels]{multibib}
\usepackage[ManyBibs]{currvita}
%\usepackage[dvips]{color}
%\usepackage[T2A]{fontenc}
%\usepackage[english,russian]{babel}
\usepackage{amsfonts,amssymb,graphicx}
%\usepackage{multicol}
\usepackage[rflt]{floatflt}
% Initialize each paper type for which you need a bibliography.
% Just a dummy parameter is necessary.
\usepackage{lastpage,fancyhdr}
\date{}
\newcites{refpaper}{foo}
\newcites{unrefpaper}{foo}
\newcites{journal}{foo}
\newcites{techreport}{foo}
\newcites{submitpaper}{foo}
\newcites{classproject}{foo}

% Better for lists with 1-2 items and short descriptions
\newenvironment{sublist}{%
	\begin{list}{}{%
		\setlength{\itemsep}{0em}\setlength{\parsep}{0em}%
		\setlength{\topsep}{0em}\setlength{\parskip}{0em}%
	}%
}%
{ \end{list} }

% Better for lists with more than 2 items and/or long descriptions
\newenvironment{subbulletlist}{%
	\begin{list}{\labelitemii}{%
		\setlength{\topsep}{\itemsep}\setlength{\parskip}{\parsep}%
	}%
}%
{ \end{list} }
%Fancy package for page enumeration and headers
\pagestyle{fancy}
%\pagestyle{empty}
%\cfoot{Page \thepage\ of \pageref{LastPage}}
%\cfoot{Page \thepage\ of 2}
\cfoot{}
\lhead{\Large \textbf {Daria Shipilina}}
\chead{{\textbf{\large \itshape Curriculum Vitae  }}}%Curriculum %Vitae 
\voffset=-1.cm
%\textheight = 670pt
%\footskip = 60pt
%\headheight=18pt
\textheight = 670pt
%\voffset=1.25cm
\headsep=26pt
\footskip = 25pt
\begin{document}
% Specify the bibtex style that you want for each paper type
\bibliographystyle{unsrt}
\bibliographystylerefpaper{unsrt}
\bibliographystyleunrefpaper{unsrt}
\bibliographystylesubmitpaper{unsrt}
\bibliographystylejournal{unsrt}
\bibliographystyletechreport{unsrt}
\bibliographystyleclassproject{ieeetr}

% We'll use this length to change the defaults in some of our lists.
\newlength{\oldcvlabelwidth}
\renewcommand*{\cvbibname}{}

% This is what will appear at the very top of your CV.



\begin{cv}%{Maksym Serbyn\\{\large \itshape Curriculum Vitae}}
% Items have more vertical space between them than line breaks.



\begin{cvlist}{Contact}
    \item Am Campus 1\\
     Klosterneuburg, Austria 3400\\
      phone: +43-677-623-45763\\
      e-mail: daria.shipilina@ist.ac.at%daria.sh@berkeley.edu
\end{cvlist}



\begin{cvlist}{Education}
\item[2011 -- 2014]   \emph{Ph.D. in Biology}\\
       Department of Vertebrate Zoology,  Moscow State University, Moscow, Russia
        \begin{sublist}
     		\item Thesis:~\textit{Hybridization between siberian and east-european chiffchaffs: morphological,  bioacoustical and genetical aspects} (advisor Dr. I. Marova) %\& Dr. N. Backst\"orm)
	       \end{sublist}
	\item[2005 -- 2010] \emph{M.S. in Biology}\\
             Department of Vertebrate Zoology,  Moscow State University, Moscow, Russia
        \begin{sublist}
       % \item GPA: 3.99 out of 4.00
		\item Thesis:~\textit{Acoustic and genetical differentiation in a wide zone of hybridization between siberian and east-european chiffchaffs} (advisor Dr. I. Marova)
	       \end{sublist}
\end{cvlist}


%\begin{cvlist}{Current position}
%\item[2017-present] \textbf{Research Trainee}, Institute of Science and Technology Austria (Barton Lab), Austria%\\[-20pt]
%\begin{itemize}%\itemsep = -1pt
%\item Supported by Kungliga Fysiografiska Sällskapet i Lund (Nilsson-Ehle Donations), Helge Ax:son Johnsons Foundation and the Lars Hierta Memorial Foundation.
%\item Obtained the first genome-wide scan for Chiffchaff, \textit{Phylloscopus collybita})
%\item Implemented data processing from raw genome reads to a set of SNPs %(and optimized) 
%\item Used SNP set to quantify genetics differences between populations 
%\item Analyzed and visualized population genetics patterns using R (hierfstat, smartPCA, ggplot), Structure 
%\end{itemize}

%\end{cvlist}


\begin{cvlist}{Research experience} %{Previous positions}
\item[2018 -- present] \textbf{Postdoctorate Researcher}, Prof. Nicholas Barton Group \\
\textit{Institute of Science and Technology Austria (IST)}, Austria%\\[-20pt]
%\item[2014 -- 2018] \textbf{Research Affiliate}, Department of Evolutionary Biology (Backstr\"om Lab),\\ \emph{Uppsala \mbox{University}}, Sweden%\\[-20pt]
%\begin{itemize}%\itemsep = -1pt
\item[2014 -- 2018] \textbf{Maternity leave} with Marfa Serbyn (02.07.2011) and Tikhon Serbyn (10.05.2013)
\small{Note: Due to birth of my children and then official maternity leave, my teaching and conference activities were limited
in 2011-2017.} 
%\item Supported by Kungliga Fysiografiska Sällskapet i Lund (Nilsson-Ehle Donations), Helge Ax:son Johnsons Foundation and the Lars Hierta Memorial Foundation.
%\item Obtained the first genome-wide scan for Chiffchaff, \textit{Phylloscopus collybita})
%\item Implemented data processing from raw genome reads to a set of SNPs %(and optimized) 
%\item Used SNP set to quantify genetics differences between populations 
%\item Analyzed and visualized population genetics patterns using R (hierfstat, smartPCA, ggplot), Structure 
%\end{itemize}
 \item[2010 -- 2011] \textbf{Academic visitor},  Department of Organismic and Evolutionary Biology (Edwards Lab) \\ %Museum of Comparative Zoology Labs, \textbf{Research Trainee}
 \textit{Harvard University}, USA%\\[-20pt]
    % \begin{itemize}%\itemsep = -1pt
     %\item \textbf{\color{red}DESCRIBE what you did, to emphasize your relevant skills, not scientific results!}
     %\item Obtained cis-regulatory sequences for genes involved in disease response using plasmid library
     % \item Discovered an evidence for a recent evolution in a response to pathogen in House Finch \textit{C. mexicanus}
%\end{itemize}
 \item[2007 -- 2014] \textbf{Research Assistant}, Department of Vertebrate Zoology\\ \textit{Moscow State University}, Russia%\\[-20pt]
 %Ecology and evolution of terrestrial Vertebrates group
    %\begin{itemize}\itemsep = -1pt
 % \item Collected DNA samples for 200 birds in independent field expeditions
 %\item Independently performed boiacoustic, morphological and genetic analysis %(mitochondrial DNA)
% \item  Proved hybridization and gene exchange in two cryptic bird species %using a data set of more then 200 individuals 
%\item  Conducted a complex research project, including independent sampling expeditions, boiacoustic, morphological and genetic analysis
% \item   Studied mechanisms of reproductive isolation between taxons 
% \end{itemize}
   


\end{cvlist}


%5\begin{cvlist}{Research Experience}
%\item[2014-present] Collaboration with Dr. Niclas Backstrom, Department of Evolutionary Biology, Evolutionary Biology Centre, Uppsala University, Sweden
%We studied genome-wide genetic variation in chiffchaffs using next generation sequencing methods. I pre-processed DNA samples of over 40 individuals for NGS.  I implemented workflow which enabled us to process data from raw genome reads to a set of SNPs. Using discovered SNPs we analyzed the population structure, and quantified degree of hybridization between subspecies. 
% \item[2010 -- 2011] Visitor researcher at  Prof. Edwards laboratory, Department of Organismic and Evolutionary Biology, %Museum of Comparative Zoology Labs,
% \textit{Harvard University}, Cambridge, MA.
%    \item[2007 -- 2014] Research Assistant, Ecology and evolution of terrestrial Vertebrates group and 
%    Laboratory of  Molecular-Genetic Methods in Zoology, \textit{Moscow State University}, Moscow, Russia
  %I studied ethological, ecological and genetic mechanisms of evolution and population differentiation of birds (family \textit{Sylviidae}), combining field, bioacoustic and genetic approaches. I participated in 5 field trip to collect materials. I performed acoustic analysis of a recorded songs, analyzed mitochondrial DNA and morphological data. My work resulted in a comprehensive description of hybridization zone between east-european and siberian chiffchaffs. 
%I investigated candidate loci responsible for disease adaptation in house finches. I analyzed rapidly evolving parts of genome. Using subsequent candidate loci resequencing, we found an evidence for a recent evolution in a response to pathogen. 
%\item[August 2009 ] Summer field school at Cranberry Lake Biological Station, \emph{State University of New York}
        %I studied white-throated sparrows in the research group headed by Prof. Tuttle. My work included capturing, nest monitoring, and processing blood samples.
%\end{cvlist}


%\begin{cvlist}{Qualifications and Skills}
 %   \item \textbf{Methods of scientific analysis}
%	\begin{sublist}
%\item \textbf{Genetics and bioinformatics}: DNA quality control and NGS library preparation, Bioinformatic tools: BWA, samtools, GATK, genomic databases: GenBank, ensemble; BLAST
%\item \textbf{Data analysis/Computer skills}: MS Office (Word, Exel, PowerPoint, Access), Adobe Photoshop and Illustrator, R (including hierfstat, smartPCA, ggplot), 
%\item \textbf{Animal models}: Handling of various vertebrate and invertebrate animals, taking blood samples, capturing in nature, maintaining a study collection.
%\item \textbf{Molecular biology}: Manual and automatic DNA extraction from blood, tissue samples and from plasmid/BAC, real time PCR, primers design, DNA restriction digest, agarose and polyacrylamide electrophoresis, 
%\item \textbf{Field experience}: sound/song recording, mist netting, operation of GPS devices, methods of rodent mammals capturing, museum sample preparation, geobotanical analysis
%\item \textbf{Cell biology}: Gene library (plasmid) construction, multiplication (Q-bot clone-picking robot), transformation in situ hybridization, radioactive isotope analysis, working with bacteria cultures manually and using robotic systems, 
%\item \textbf{Animal models}: Handling of various vertebrate and invertebrate animals, taking blood samples, capturing in nature, maintaining a study collection.

\begin{cvlist}{Grants and awards}
 \item[2018] \textbf{Erasmus+ Staff Mobility Grant}\\
 Collaboraive project and training at \textit{John Innes Centre}, Norwich, UK 
\item[2013] \textbf{Royal Swedish Academy of Science} (KVA) Project Grant \\
 \textit{(Together with Prof. Niclas Backstr\"om)} \\
 Supplementary funding for genomic sequensing %SEK 100,000 The genetics of reproductive isolation in an avian hybrid zone \textit{(Together with Prof. Niclas Backstr\"om)} SEK 100,000 Project Grant (ID\#: FOA13H-088)
\item[2012] \textbf{Helge Axelson Johnson's Foundation}\\
 \textit{(Together with Prof. Niclas Backstr\"om)} \\
 Supplementary funding for genomic sequensing
 %The genetics of reproductive isolation in a chiffchaff (\textit{Phylloscopus collybita} sp.) hybrid zone. %SEK 45,000
\item[2011] \textbf{Lars Hiertas Memory Foundation} \\
 \textit{(Together with Prof. Niclas Backstr\"om)} \\
 Supplementary funding for genomic sequensing
%The genetics of reproductive isolation in chiffchaff (Phylloscopus collybita sp.). %SEK 20,000
\item[2011] \textbf{Nilsson-Ehle-Foundation}, Royal Physiographic Society in Lund \\
 \textit{(Together with Prof. Niclas Backstr\"om)} \\
 Supplementary funding for genomic sequensing
%The genetics of reproductive isolation in a chiffchaff (Phylloscopus collybita sp.) hybrid zone. SEK 83,000
\item[2009] \textbf{Russian Fund for Basic Research} Travel Grant\\
Attended 8th Conference of European Ornithological Union, August 2009
\end{cvlist}


%The resulting labeled DNA can serve as a sensitive hybridization probe for screening gene libraries The Typhoon instrument is a variable-mode imager that produces digital images of radioactive 
%•	Plasmid/BAC DNA isolation protocol


%•	



%•	
%Teaching
%•	Curriculum design and development, syllabus construction, lecturing Training new students in basic molecular biology methods

%Analytical skills
%•	MS Office (Word, Excel, and Powerpoint), Adobe Photoshop, Sigma Plot, Chem Draw, and PubMed search. BWA, Picard, GATK?-
%•	Data analysis: sequencing databases, BLAST?- Data analysis: R, StatSoft, Microsoft Exel, STRUCTURE, EigenSoft
%Analysis of genetic structure of the population of non-model organism (Phylloscopus collybita)?- Working with raw NGS data:: DNA extraction, quality control including gel electrophoresis, SDS-PAGE, library preparation?- 
%\item \textbf{Molecular biology}: Manual and automatic DNA extraction from blood, tissue samples and from plasmid/BAC, real time PCR, primers design, DNA restriction digest, agarose and polyacrylamide electrophoresis, 
%	\end{sublist}
	
%\item \textbf{Computer skills}
%\begin{sublist}
%		\item MS Office (Word, Exel, PowerPoint, Access), Adobe Photoshop, Illustrator (basic).
%\end{sublist}

	
%\setlength{\cvlabelwidth}{\oldcvlabelwidth}

%\end{cvlist}

%\begin{cvlist}{Leadership experience}
%\item[2008] Participated in organizing several Greenpeace Russia programs;  taught ornithology classes at ``Kids for Forest Restoration'' summer camp 
%\end{cvlist}

%\begin{cvlist}{Environmental Education and Teaching}
      % \item[2015-2016] Teaching naturalists classes for youth, including botany, zoology, ecology and environmental education at ``Krugozor'' after-school program, Berkeley, Curriculum design and development, syllabus construction, lecturing
	%\item[2009-2011] Training undergraduate students in field and molecular biology methods at  Moscow State University and Harvard University
	%\item[2008] Participated in organizing several Greenpeace Russia programs;  taught ornithology classes at ``Kids for Forest Restoration'' summer camp 
%\end{cvlist}



\newpage


\begin{cvlist}{Teaching experience and student supervision}
\item[2018] Teaching Assistant, IST Biology Core course
\item[2011] Co-supervision of undergraduate summer project \\
Museum of Comparative Zoology, Harvard University 
\item[2009 -- 2010] Training undergraduate students in molecular biology methods and field work,\\ Laboratory of Molecular Methods in Zoology, Moscow State University
\end{cvlist}

\begin{cvlist}{Conference contributions} %{Selected conference presentations}
%\item 2013  -----  9th Conference for European Ornithological Union, \textit{Norwich}. 
%\item  2012  ----- Evolution Conference,  \textit{Ottawa}.
%\item  2012  ----- V All-Russian Conference on animal bahavior, \textit{Moscow}
%\item  2011  -----8th Conference for European Ornithological Union, \textit{Riga}. 
%\item  2011  -----American Genenetics Assosiation conference, \textit{Guanajuato, Mexico}.
%\item  2010  -----10th International Ornithological Congress, \textit{Campos do Jordao}. 
%\item  2010  ----- XIII International Ornithological Conference of Northern Eurasia, \textit{Orenburg}.
 \item[2010] Current problems of ecology and conservation, RUDN, {Moscow}.
\item[2010]  XVIII International young scientist conference ``Lomonosov'',{Moscow}
\item[2009]  7th Conference for European Ornithological Union, {Z\"urich}
\item[2009] XXVI International  young scientist conference ``Lomonosov'', {Moscow}

\end{cvlist}

\begin{cvlist}{Outreach}
\item[2018] Selected Topics in Evolutionary Biology for High School Students, workshop\\
Talk: ``Birdsong in evolutionary studies''
\item[2018] IST Bioinformatics Workshop for Master Students\\
Talk: ``Bioinformatic tools and approaches in application to \textit{Antirrhinum} project''
\item[2015 -- 2016 ] Evolutionary and Field Biology for Youth Course\\
After-school program ``Krugozor'', Berkeley
\item[2008] Greenpeace Russia Summer Program\\
Course: ``Bird identification for conservation research''\\
Talk: ``Conservation genetics for beginners''
%``Kids for Forest Restoration'':
\end{cvlist}

%\begin{cvlist}{Distinctions and awards}
%\item[2009] \emph{Best plenary presentation award}, XXVI International  young scientist conference ``Lomonosov'', Moscow
%\end{cvlist}



\begin{cvlist}{Extras}
     \item [2014 -- present] \textbf{Contributing member}\\
     EBCC Atlas of European Breeding Birds (Russia)
	
\item [2017] Reviewer for \textit{PeerJ}
\item[2009] \textbf{Best plenary presentation award}\\
XXVI International  young scientist conference ``Lomonosov'', Moscow
\end{cvlist}


% These commands will make our papers be indented after the first line
% to make the author's name stand out more.  Also, we reduce the
% indentation of the paper type titles (e.g., Journal Papers,
% Conference Paper) compared to regular cvlist items.

%\setlength{\oldcvlabelwidth}{\cvlabelwidth}
%\setlength{\cvlabelwidth}{6em}%1
%\renewcommand*{\bibindent}{1.5em}


% Do bibaddtoleftmargin instead of biblabelsep if you're
% not using manybib or open bib
%\renewcommand*{\bibaddtoleftmargin}{1.5em}
%\renewcommand*{\biblabelsep}{1.5em}

% To update, before doing latexmk, run:
% bibtex journal
% bibtex refpaper
% bibtex unrefpaper
% bibtex techreport





\newpage
\ \\[-60pt]
\begin{cvlist}{Publications}
\item
\end{cvlist}

%\begin{itemize}
%\item[]

%\textbf{\quad Papers under review}
%\end{itemize}

	%\begin{enumerate}%{sublist}{enumerate}
	%\item  V. Talla, F. Kalsoom, \textbf{D. Shipilina}, I. Marova, N. Backstr\"om
	%"Genome-wide patterns of genetic diversity and differentiation in European and Siberian chiffchaff (\textit{Phylloscopus collybita abietinus / P. tristis})"
	%\end{enumerate}%{sublist}

\textbf{\quad Book chapters}
	
	%\begin{enumerate}%{sublist}{enumerate}	
		%\item  Marova, I., \textbf{Shipilina, D.}, Fedorov, V., Ivanitskii, V., \textbf{Shipilina, D.}\\
          %  Siberian and East European chiffchaffs: geographical distribution, morphological features, vocalization, phenomenon of mixed singing, and evidences of hybridization in sympatry zone, in: "El mosquitero ib{\'e}rico", Grupo Iberico de Anillamiento, L{\'e}on, \textbf{2013}, 119--139.
            \begin{enumerate}%{sublist}{enumerate}	
		\item  Marova, I., \textbf{Shipilina, D.}, Fedorov, V., Ivanitskii, V., \textbf{2013}\\
            Siberian and East European chiffchaffs: geographical distribution, morphological features, vocalization, phenomenon of mixed singing, and evidences of hybridization in sympatry zone, in: "El mosquitero ib{\'e}rico": 119--139.
\end{enumerate}%{sublist}
	
	% With multibib, each cite command has the type after it.
\textbf{\quad Journal Papers}
	
	\begin{enumerate}%{sublist}
	%\item  V. Talla, F. Kalsoom, \textbf{D. Shipilina}, I. Marova, N. Backstr\"om
	%Heterogeneous Patterns of Genetic Diversity and Differentiation in European and Siberian Chiffchaff (\textit{Phylloscopus collybita abietinus / P. tristis}), \textit{G3}, \textbf{2017}, in print.
	\item Talla, V., Kalsoom, F., \textbf{Shipilina, D.}, Marova, I., and Backström, N. \textbf{2017}. 
	Heterogeneous patterns of genetic diversity and differentiation in European and Siberian chiffchaff (Phylloscopus collybita abietinus / P. tristis). G3: Genes, Genomes, Genetics 7: 3983-3998.
	%\item  \textbf{D. Shipilina}, M. Serbyn, V. Ivanitskii, I. Marova, N. Backstr\"om\\
	       %  "Patterns of genetic, phenotypic and acoustic variation across a chiffchaff (\textit{Phylloscopus collybita abietinus/tristis}) hybrid zone",  \textit{Ecology and evolution}, \textbf{2017}, Vol.~7., 2169--2180.
	 \item  \textbf{Shipilina, D.}, Serbyn, M., Ivanitskii, V., Marova, I., Backström, N. \textbf{2017}\\
	        Patterns of genetic, phenotypic and acoustic variation across a chiffchaff (\textit{Phylloscopus collybita abietinus/tristis}) hybrid zone. Ecology and evolution 7: 2169--2180.        
	%\item  I.Marova, \textbf{D.Shipilina}, V.Fedorov, V.Alekseev, V.Ivanitskii\\
	        % "Interaction between Common and Siberian Chiffchaff in a contact zone",  \textit{Ornis Fennica},\\ \textbf{2017}, Vol.~94.,~66--81.
	 \item  Marova, I., \textbf{Shipilina, D.}, Fedorov, V., Alekseev, V., Ivanitskii V. \textbf{2017}\\
	         "Interaction between Common and Siberian Chiffchaff in a contact zone",  \textit{Ornis Fennica},\\ \textbf{2017}, Vol.~94.,~66--81.
	         
	%\item I.Marova, \textbf{D.Shipilina}, V.Alekseev, V.Fedorov, V.Ivanitskii\\
		%"Hybridisation of Eastern European and Siberian chiffchaffs ( \textit{Phylloscopus collybita abietinus - P.c. tristis}): complex analysis of the sympatry zone on Southern Ural",  		\textit{Ornitologia}, \textbf{2016}, Vol.~40., 27--45. %[In Russian]
		\item Marova, I., \textbf{Shipilina, D.}, Alekseev, V., Fedorov, V., Ivanitskii V.\\textbf{2016}\
		Hybridisation of Eastern European and Siberian chiffchaffs (\textit{Phylloscopus collybita abietinus - P.c. tristis}): complex analysis of the sympatry zone on Southern Ural. Ornitologia 40: 27--45. %[In Russian]

 	%\item  N. Selivanova, \textbf{D. Shipilina}, A.Yestafief, I. Marova\\
            %	``Intrerspecific variability of the Chiffchaff (\textit{Phylloscopus collybita}, Sylviidae, Aves) in a sympatry zone of Siberian and Eastern European form in Komi republic  (morpholgical, bioacoustical and genetical analysis)'', \textit{Bulletin of Moscow society for Natural History. Division of biology}, \textbf{2014}, Vol. 119, n.~1, 3--16. %[In Russian]
          \item  N. Selivanova, \textbf{Shipilina, D.}, A.Yestafief, Marova, I. \textbf{2014}\\
            	Intrerspecific variability of the Chiffchaff (\textit{Phylloscopus collybita}, Sylviidae, Aves) in a sympatry zone of Siberian and Eastern European form in Komi republic  (morpholgical, bioacoustical and genetical analysis). Bulletin of Moscow Soc. for Nat. Hist. 119, n.~1:3--16. %[In Russian]
  
	%\item I.Marova, \textbf{D.Shipilina}, V.Alekseev, V.Fedorov, V.Ivanitskii\\
		%"Zone of the secondary contact between Eastern European and Siberian chiffchaffs \textit{(Phylloscopus collybita abietinus - Ph. c.tristis)} on Southern Ural: proofs of the hybridisation", \textit{Proceedings of Southern Ural State Natural Reserve}, \textbf{2014}, Vol. 2, 95--117 %[In Russian]
	\item Marova, I., \textbf{Shipilina, D.}, Alekseev, V., Fedorov, V., Ivanitskii, V. \textbf{2014}\\
		Zone of the secondary contact between Eastern European and Siberian chiffchaffs \textit{(Phylloscopus collybita abietinus - Ph. c.tristis)} on Southern Ural: proofs of the hybridisation. Proc. of Southern Ural State Reserve 2: 95--117 %[In Russian]

	%\item  N. Backstr{\"o}m, \textbf{D. Shipilina}, M. Blom, S. Edwards\\
	         %  ``Cis-regulatory sequence variation and association with Mycoplasma load in natural populations of the house finch (\textit{Carpodacus mexicanus})'', \textit{Ecology and evolution}, \textbf{2013}, Vol.~3., n.~3, 655--666.
	\item  Backstr{\"o}m, N., \textbf{Shipilina, D.}, Blom, M., Edwards, S. \textbf{2013}\\
	           Cis-regulatory sequence variation and association with Mycoplasma load in natural populations of the house finch (\textit{Carpodacus mexicanus}). Ecology and evolution 3., n.~3: 655--666.
    
          %\item  \textbf{D. Shipilina}, I. Marova\\
          	%``Habitats, population structure and diversity of the song of the Caucasus Chiffchaff (\textit{Phylloscopus} [\textit{sindianus}] \textit{lorenzii}) on the Northern Caucasus'', \textit{Ornitologia}, \textbf{2013}, Vol.~38, 54--63 %[In Russian]

 \item \textbf{Shipilina, D.}, Marova, I. \textbf{2013}\\
          	Habitats, population structure and diversity of the song of the Caucasus Chiffchaff (\textit{Phylloscopus} [\textit{sindianus}] \textit{lorenzii}) on the Northern Caucasus. Ornitologia 38: 54--63 %[In Russian]
	
	%\item Y. Deart, \textbf{D.Shipilina}
	%"Anatomy of the seedling of \textit{Tilia cordata}", in: "Morphology and anatomy of flora of the Usman forest (Voronezh region)", MAKS Press, \textbf{2010}, 33-37.
	
	%\item  I. Marova, V. Fedorov, \textbf{D. Shipilina}, V. Alekseev\\
%``Genetic and Vocal Differentiation in Hybrid Zones of Passerine Birds: Siberian and European Chiffchaffs (\textit{Phylloscopus} [\textit{collybita}] \textit{tristis} and \textit{Ph. [c.] abietinus}) in the Southern Urals'', \textit{Doklady Biological Sciences}, \textbf{2009}, \textit{427}, 384--386.

\item  Marova, I., Fedorov, V., \textbf{Shipilina, D.}, Alekseev, V. \textbf{2009}\\
Genetic and Vocal Differentiation in Hybrid Zones of Passerine Birds: Siberian and European Chiffchaffs (\textit{Phylloscopus} [\textit{collybita}] \textit{tristis} and \textit{Ph. [c.] abietinus}) in the Southern Urals. Doklady Biological Sciences 427: 384--386.

	%\item  I. Marova, V. Fedorov, \textbf{D. Shipilina}, V. Alekseev\\
          %  ``Ornithofauna of western slopes of forest covered mountains of Southern Ural'', \emph{Materials on birds distribution on the Ural mountains, around the Ural region and Western Siberia}, \textbf{2008}, 62--69

\item  Marova, I., Fedorov, V., \textbf{Shipilina, D.}, Alekseev, V. \textbf{2008}\\
            Ornithofauna of western slopes of forest covered mountains of Southern Ural. Materials on birds distribution on the Ural mountains, around the Ural region and Western Siberia: 62--69

\end{enumerate}%{sublist}











%\begin{cvlist}{Teaching Experience}

%\end{cvlist}

\setlength{\oldcvlabelwidth}{\cvlabelwidth}
\setlength{\cvlabelwidth}{6em}





% Reset the label indentation
%\newpage
\setlength{\cvlabelwidth}{\oldcvlabelwidth}


%\begin{cvlist}{Personal Information}
%    \item Date of birth:  August 21, 1988\\
          %Citizenship:  Russia \\
 %         Languages: English, Russian.
%\end{cvlist}
%\begin{cvlist}{Personal Information}
 %   \item Date of birth:  January 30th, 1986\\
 %         Citizenship:  Ukraine \\
 %         Languages: English (fluent), Russian (native), Ukrainian (native), German (basic)
%\end{cvlist}

%\begin{cvlist}{Personal Interests}
%\item  Reading, Travelling, Music
%\end{cvlist}
%\footskip = -10pt
%%%%%%%%%%%%%%%%%%%%%%%%%%%%%%%%%%%%%%%%%%%%%%%%%%%%%%%%%%%%%%%%%%%%%%%%%%%%%%%%%%%%%%%%%
%%%%%%%%%%%%%%%%%%%%%%%%%%%%%%%%%%%%%%%%%%%%%%%%%%%%%%%%%%%%%%%%%%%%%%%%%%%%%%%%%%%%%%%%%
\end{cv}
\end{document}

